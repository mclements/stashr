\documentclass{article}

%%\VignetteIndexEntry{The stashR Package}
%%\VignetteDepends{filehash,tools,stashR}

\usepackage{charter}
\usepackage{courier}
\usepackage[noae]{Sweave}
\usepackage[margin=1in]{geometry}
\usepackage{natbib}

\title{Interacting with local and remote data repositories\\ 
using the \textbf{stashR} package for R}

\author{Sandrah P. Eckel and Roger D. Peng \\
\textit{Department of
Biostatistics}\\\textit{Johns Hopkins Bloomberg School of Public Health}}

\date{}

\newcommand{\pkg}{\textbf}
\newcommand{\code}{\texttt}

\begin{document}

\maketitle

\begin{abstract}
The \pkg{stashR} package (a Set of Tools for Administering SHared
Repositories) for R implements a simple key-value style database
where character string keys are associated with data values. The
key-value databases can be either stored locally on the user's
computer or accessed remotely via the Internet. Methods specific to
the \pkg{stashR} package allow users to share data repositories or
access previously created remote data repositories. In particular,
methods are available for the S4 classes `localDB' and `remoteDB'
to insert, retrieve, or delete data from the database as well as to
synchronize local copies of the data to the remote version of the
database. Users efficiently access information from a remote
database by retrieving only the data files indexed by
user-specified keys and caching this data in a local copy of the
remote database. The local and remote counterparts of the \pkg{stashR}
package offer the potential to enhance reproducible research by
allowing users of \pkg{Sweave} to cache their R computations for a
research paper in a `localDB' database. This database can then be
stored on the Internet as a `remoteDB' database. When readers of
the research paper wish to reproduce the computations involved in
creating a specific figure or calculating a specific numeric value,
they can access the `remoteDB' database and obtain the R objects
involved in the computation.


\end{abstract}


\section{Overview and Motivation}

The R package \pkg{filehash} addresses the issue of how to work interactively 
with large data sets that cannot be loaded into R as a single object due to 
limitations in physical memory size. The \pkg{stashR} R package extends the 
\pkg{filehash} package to local and remote databases. As is, the package 
\pkg{stashR} can be used to create a local `localDB'
key-value database where data files are indexed by character string keys. The
\pkg{stashR} package can also be used by individuals to download
data from a `remoteDB' key-value database stored remotely on the 
internet. This dual functionality of the \pkg{stashR} package
offers potential applications to future work in creating research documents
that satisfy the demands of reproducible research.

\subsection{Contributions of the \pkg{stashR} Package}
The \pkg{stashR} package adds important functionalities to data handling in R.
These contributions include:
\begin{itemize}
    \item the ability to access remote databases efficiently
    \item a set of tools for creating a local database to export to a remote
        locale
    \item a tool for synchronizing local copies of a database to the remote 
        version
    \item an abstract interface consisting of 6 methods for local and remote 
        databases.
\end{itemize}    


\section{Design Rationale}


\subsection{Repository Layout} \label{layout}

The repository is composed of a root directory containing a data directory and
a text file,`keys', that lists of each of the character keys corresponding
to a data file in the data directory. The data directory contains compressed
data files labelled according to their corresponding character key. Each data file
has a corresponding `.SIG' text file that lists the 32-byte MD5 checksum from 
running \code{md5sum()} on the data file (see the R package \pkg{tools} for more 
details) and the data file's identifying character key. We will discuss the role 
of the `.SIG' files in greater detail in Sections~\ref{synch-rationale} 
and~\ref{synch-interface}. 
 
\subsection{Remote vs. Local}

The \pkg{stashR} package is designed for interacting with
both local and remote data repositories. Each of the main user
interface functions in \pkg{stashR} is a generic function 
with specific methods defined for repository objects of class
`localDB', the local version of a key-value database, and for
objects of class `remoteDB', the remote version of a key-value
database. When interacting with a `localDB' data repository, 
the user can insert, fetch, delete and list keys of the available 
data files. When interacting with a `remoteDB' data 
repository, the user creates a local copy of the repository that
contains only the desired data files from the remote repository. The 
user interface functions for the `remoteDB'repository 
are similar to those for the 'localDB' repository. When 
the user fetches data from a remote repository, it is either accessed 
using the local cache, or downloaded from the remote repository if it has not
previously been downloaded. The \pkg{stashR} package also
has a feature to synchronize the local copy of a repository to the
remote repository.


\subsection{Caching and Synchronization} \label{synch-rationale}

The \pkg{stashR} package allows users
to cache or access cached data in a `localDB' or
`remoteDB' repository. A key feature of the
\pkg{stashR} package, is the ability for a user to download
desired data from a 'remoteDB' repository in a local directory
and, at a later date, synchronize their locally cached data to the
data in the remote `remoteDB' data repository. The synchronization
feature allows a user to efficiently maintain an up-to-date local cache of 
remotely stored data by downloading updated versions of the remote
data files only when needed. 

As noted in Section~\ref{layout}, each data file has a corresponding
`.SIG' file that contains the MD5 checksum of the data file along with 
the key indexing the data file. The MD5 checksum is 
theoretically a nearly unique character string that identifies a file. If a 
small change is made to a file, its corresponding MD5 checksum will change 
dramatically. Sychronization in the package \pkg{stashR} is acheived 
by comparing the MD5 checksum in the `.SIG' file corresponding to the local 
copy of the data and the MD5 checksum in the`.SIG' file corresponding to the 
remote data file. If a this data file has been modified on the remote data
repository, the MD5 checksums will not match, and the new data file and 
'.SIG' file will be downloaded to the local copy of the repository to 
synchronize the local copy of the data to the remote version of the data. 


\section{Interface}

\subsection{Creating a Local remoteDB database}

There are two steps to creating a `localDB' or a `remoteDB'
object. The first step is to call \code{new("localDB", dir = dir, name = name)} 
or \code{new("remoteDB", dir = dir, url = url, name = name)} where 
`dir' is a character string specifying the local directory in which to 
create the new `localDB' repository of the local copy of the 
`remoteDB' repository. The `url' argument is unique to objects of the 
`remoteDB' class and it specifies as a character string the url 
of the root directory of the remote key-value database. The `name' 
argument is a character string specifying the label that will be associated with
the `localDB' or `remoteDB' repository.  

The next step in creating a `localDB' or a `remoteDB' 
database is to call the \code{dbCreate} function.  The function \code{dbCreate} 
requires only the `localDB' or `remoteDB' object as an argument. The
information for where to build the local repository is stored in the `dir' slot
of the `localDB' or `remoteDB' object.


\subsection{Accessing a remoteDB database}

The user-end interfaces to `localDB' or `remoteDB' databases 
are of the functions \code{dbFetch}, \code{dbInsert}, \code{dbList}, 
\code{dbExists}, \code{dbDelete} and \code{dbSync}.  Each of these functions 
is generic and has a specific method for objects of the 
`remoteDB' and `localDB' classes (with the exception of \code{dbSync}, 
as we shall see in Section~\ref{synch-interface}). The first argument for any of the 
above functions is an object of class `remoteDB' or `localDB'.

\subsubsection{\code{dbFetch}}

The function \code{dbFetch(db = "localDB" or "remoteDB", 
key = "character")} takes two arguments. The first argument is either a 
`localDB' or `remoteDB' object. The second argument is a 
character string key indexing a data file. 

For objects of the class `remoteDB', \code{dbFetch} first checks to 
see if the provided key's data file and .SIG file exist 
in the local copy of the `remoteDB' repository. If the data and .SIG files
indexed by the key do not exist, then \code{dbFetch} downloads the 
two files from the remote repository to the local copy and reads the data file. 
If the data and .SIG file do exist in the local copy of the repository, then
\code{dbFetch} compares the MD5 sumcheck stored in the .SIG file from the local 
repository to the MD5 sumcheck stored in the .SIG file in the remote 
repository. If the MD5 sumchecks are the same, then \code{dbFetch} reads 
the file from the local repository. Otherwise, \code{dbFetch} downloads 
the updated version of the data and .SIG files from the remote repository
and reads the data file. The object associated with the key, that was 
stored in the corresponding data file, is returned by \code{dbFetch.}

Similarly, for objects of the class `localDB', \code{dbFetch}
checks if the provided character value key's data file and `.SIG' file exist 
in the local repository. If the files exist, then \code{dbFetch} reads the
data file from the local directory and returns the R object stored in the
data file. If the corresponding files do not exist, then \code{dbFetch} 
returns an error. 

\subsubsection{\code{dbInsert}}

The function \code{dbInsert(db = "localDB" or
"remoteDB", key = "character", value= "ANY", overwrite = TRUE)} 
takes four arguments. The first argument, like any of the other user-end
interfaces is either a `localDB' or `remoteDB' object.
The second argument is a character string key indexing the file
that will be created to store the object indicated by the `value'
argument. The third argument, value, is any R object that the user 
wishes to store in the repository. 

Calling \code{dbInsert} on a `remoteDB' object returns an
error message. Thus the user cannot write to a remote repository 
or write to his or her local copy of the remote repository, which would
make the two versions of the repository out of sync. 

On the other hand, calling \code{dbInsert} on a `localDB' object 
writes the value to a data file corresponding to the specified key within 
the local data directory. Also, \code{dbInsert} appends the specified key 
to the end of the `keys' file if the key is not already included in the 
`keys' file. The fourth argument of the \code{dbInsert} function allows the
user to specify whether or not they will allow \code{dbInsert} to overwrite
a pre-existing file with the same key. The defalt is set to `TRUE' so that 
\code{dbInsert} will overwrite a pre-existing file indexed by the same
key as the file that the user is trying to insert with \code{dbInsert}.

\subsubsection{\code{dbList}}

The function \code{dbList(db = "localDB")} or 
\code{dbList(db = "remoteDB", save = FALSE)} takes a `localDB' 
or `remoteDB' object as its argument. For `remoteDB' objects, 
there is also an option to save the `keys' file from the remote repository
to the analogous location in the local copy of the repository. For both 
classes of objects, \code{dbList} reads the character string key values 
stored in the `keys' file of the repository and returns a vector of the 
keys.

\subsubsection{\code{dbExists}}

In general terms, the function \code{dbExists} allows a user to determine which 
elements of a vector of character string keys are contained in the 
repository. The function \code{dbExists(db = "localDB" or 
"remoteDB", key = key)} has a second argument, `key', which takes
a vector of character strings. The logical vector returned by \code{dbExists}
is of the same length as the vector of character keys.

For both objects of class `remoteDB' and objects of class `localDB', 
\code{dbExists} returns TRUE for each key that indexes a data file contained 
the repository (as indicated in the `keys' file of the repository).
If a key in the vector of keys specified as the key argument to \code{dbExists}
indexes a file that is not contained `keys' file of the repository,
\code{dbExists} returns FALSE in the corresponding position of the output vector
of logical values.


\subsubsection{\code{dbDelete}}

The function \code{dbDelete} allows a user to delete both the data and `.SIG' file
indexed by a particular key from the repository. The function call is 
\code{dbDelete(db = "remoteDB" or "localDB", key = "character")}. 
Calling \code{dbDelete} on a `remoteDB' object returns an error message 
since the user does not have access to the remote repository to delete the 
specified files. On the other hand, calling \code{dbDelete} on a `localDB' 
object results in the deletion of the specified data and `.SIG' file from 
the data directory of the local repository. The specified key is also 
deleted from the `keys' file in the top-level directory of the repository.  




\subsubsection{\code{dbSync}}\label{synch-interface}
The \pkg{stashR} function for synchronizing local copies
of data stored on a remote repository is the generic function
\code{dbSync}. Currently, \code{dbSync} only has a method for
objects of the `remoteDB' class because one would only need
to synchronize a local copy of a remote database. The
\code{dbSync} function takes as arguments a `remoteDB'
object and a (possibly null) character vector of keys, called
`key'. If the `key' vector contains a character string key that
corresponds to a data file that has not yet been downloaded to the
local copy of the repository, \code{dbSync} returns an error
message. If the `key' vector is null, then \code{dbSync} obtains a
list of the data files that have been locally cached, checks if
these data files have changed on the remote repository, and then
updates the necessary data files. Similarly, if the `key' vector
contains only keys for data files that have been locally
downloaded, \code{dbSync} will only check and, if necessary,
update the files specified in the `key' vector.


\section{Examples}

\subsection{Introductory Examples}

\subsubsection{Objects of the class `localDB'} 

For objects of the class `localDB', we start out by 
defining a local directory in which we will create the 
repository.

\begin{Schunk}
\begin{Sinput}
> wd <- getwd()
> dir <- file.path(wd, "localDBExample")
\end{Sinput}
\end{Schunk}

\noindent Next, we perform a two-step process to create the `localDB' object, 
which we will call `fhLocal'.
\begin{Schunk}
\begin{Sinput}
> fhLocal <- new("localDB", dir = dir, name = "localDB Example")
> dbCreate(fhLocal)
\end{Sinput}
\end{Schunk}

\noindent We now insert different types of R objects into the local repository to 
create a basic `localDB' database. Note that each time we call 
\code{dbList}, we see the keys indexing all of the data files we have inserted.

\begin{Schunk}
\begin{Sinput}
> v <- 1:10
> dbInsert(fhLocal, key = "vector", value = v, overwrite = TRUE)
> m <- matrix(1:20, 5, 4)
> dbInsert(fhLocal, key = "matrix", value = m, overwrite = TRUE)
> d <- data.frame(cbind(id = 1:5, age = c(12, 11, 15, 11, 14), 
+     sex = c(1, 1, 0, 1, 0)))
> dbInsert(fhLocal, key = "dataframe", value = d, overwrite = TRUE)
> dbList(fhLocal)
\end{Sinput}
\begin{Soutput}
[1] "vector"    "matrix"    "dataframe"
\end{Soutput}
\begin{Sinput}
> l <- list(v = v, m = m, df = d)
> dbInsert(fhLocal, key = "list", value = l, overwrite = TRUE)
> dbList(fhLocal)
\end{Sinput}
\begin{Soutput}
[1] "vector"    "matrix"    "dataframe" "list"     
\end{Soutput}
\end{Schunk}

\noindent We can fetch any of the R objects saved in our local repository.
\begin{Schunk}
\begin{Sinput}
> dbFetch(fhLocal, "dataframe")
\end{Sinput}
\begin{Soutput}
  id age sex
1  1  12   1
2  2  11   1
3  3  15   0
4  4  11   1
5  5  14   0
\end{Soutput}
\end{Schunk}

\noindent If we delete a data file from the local repository, \code{dbList} or
\code{dbExists} can be used to confirm the deletion.
\begin{Schunk}
\begin{Sinput}
> dbDelete(fhLocal, "vector")
> dbExists(fhLocal, "vector")
\end{Sinput}
\begin{Soutput}
[1] FALSE
\end{Soutput}
\begin{Sinput}
> dbList(fhLocal)
\end{Sinput}
\begin{Soutput}
[1] "matrix"    "dataframe" "list"     
\end{Soutput}
\end{Schunk}


\subsubsection{Objects of the class `remoteDB'} 

The same data used in the previous example for `localDB'
has been stored in a `remoteDB' repository on the internet at:
\begin{Schunk}
\begin{Sinput}
> myurl <- "http://www.biostat.jhsph.edu/~seckel/remoteDBExample/"
\end{Sinput}
\end{Schunk}

In this example, we will use the `remoteDB' methods for 
the \pkg{stashR} package interface functions: \code{dbFetch}, 
\code{dbList}, \code{dbExists} and \code{dbSync}. Note that we will 
not use \code{dbInsert} and \code{dbDelete} functions because these methods 
simply return error messages for `remoteDB' objects.  

Again, we start off with the two-step process of creating a `remoteDB' 
object. The local copy of the database will be located in our working
directory under a directory called `remoteDBExample'. 

\begin{Schunk}
\begin{Sinput}
> wd <- getwd()
> dir <- file.path(wd, "remoteDBExample")
> fhRemote <- new("remoteDB", url = myurl, dir = dir, name = "remoteDB Example")
> dbCreate(fhRemote)
\end{Sinput}
\end{Schunk}

\noindent When we run \code{dbList} on the `remoteDB' object, `fhRemote',
we see the same four character string keys corresponding to the data values
from the previous example. Using the \code{save = TRUE} option in \code{dbList}
saves a copy of the 'keys' file from the remote version of the database to 
the local copy of the database. The function \code{dbExists} can be used as a
shortcut, when the list of keys is long, to see which elements of a vector 
of keys are contained in the database.

\begin{Schunk}
\begin{Sinput}
> dbList(fhRemote, save = TRUE)
\end{Sinput}
\begin{Soutput}
[1] "matrix"    "dataframe" "list"      "vector"   
\end{Soutput}
\begin{Sinput}
> dbExists(fhRemote, c("vector", "array", "list", "function"))
\end{Sinput}
\begin{Soutput}
[1]  TRUE FALSE  TRUE FALSE
\end{Soutput}
\end{Schunk}

\noindent 

\noindent We can fetch any of the data values indexed by the keys resulting 
from \code{dbFetch}. Once we have downloaded a data file to the local cache,
\code{dbFetch} simply looks in the local cache for the data file rather than
downloading the file again over the internet. 

\begin{verbatim}
> dbFetch(fhRemote, "vector") 
trying URL 'http://www.biostat.jhsph.edu/~seckel/remoteDBExample//data/vector'
Content type 'text/plain; charset=UTF-8' length 59 bytes
opened URL
downloaded 59 bytes

trying URL 'http://www.biostat.jhsph.edu/~seckel/remoteDBExample//data/vector.SIG'
Content type 'text/plain; charset=UTF-8' length 42 bytes
opened URL
downloaded 42 bytes

 [1]  1  2  3  4  5  6  7  8  9 10
\end{verbatim}

\begin{verbatim}
> dbFetch(fhRemote, "matrix") 
trying URL 'http://www.biostat.jhsph.edu/~seckel/remoteDBExample//data/matrix'
Content type 'text/plain; charset=UTF-8' length 97 bytes
opened URL
downloaded 97 bytes

trying URL 'http://www.biostat.jhsph.edu/~seckel/remoteDBExample//data/matrix.SIG'
Content type 'text/plain; charset=UTF-8' length 42 bytes
opened URL
downloaded 42 bytes

     [,1] [,2] [,3] [,4]
[1,]    1    6   11   16
[2,]    2    7   12   17
[3,]    3    8   13   18
[4,]    4    9   14   19
[5,]    5   10   15   20
\end{verbatim}

\begin{Schunk}
\begin{Sinput}
> dbFetch(fhRemote, "matrix")
\end{Sinput}
\begin{Soutput}
     [,1] [,2] [,3] [,4]
[1,]    1    6   11   16
[2,]    2    7   12   17
[3,]    3    8   13   18
[4,]    4    9   14   19
[5,]    5   10   15   20
\end{Soutput}
\end{Schunk}

\noindent As mentioned previously, the function \code{dbSync} allows
a user to synchronize a local copy of a remote database to the remote version
of the database. Using the \code{key = NULL} option synchronizes all data files
in the local copy of the database, while specifying a vector of keys
synchronizes only the specified keys. 

\begin{Schunk}
\begin{Sinput}
> dbSync(fhRemote, key = NULL)
> dbSync(fhRemote, key = c("matrix", "vector"))
\end{Sinput}
\end{Schunk}


\section{Application: NMMAPS database} 

to be filled in later...
 
\section{Discussion}

The \pkg{stashR} package has been designed as a
tool for both producers and consumers of statistical documents in
the context of streamlining and enhancing reproducible research
documents created using \code{Sweave}. There is a need for future 
work linking, on the producer's end, the results of R code chunks 
of \code{Sweave} documents to a localDB database. This local 
database would then need to be transferred in an automatic way as a
remoteDB database to a repository on the internet. Ideally,
this internet repository would be a central database that all
statistical researchers could use, although individuals could 
alternatively choose to store their data on their own server. On the
consumer's end, we need to develop a method for allowing a user to
`click' on a figure or numerical result in a pdf document produced
by \code{Sweave} and then have returned to them the R objects
(stored in the remoteDB database) used in the computation of
their result of interest as well as the R code chunk that operates
on these objects. In this case, the R objects used in the
computation of each figure or numerical value would be indexed by
a key that is the name of each R code chunk in the \code{Sweave}
document.




%\bibliographystyle{asa}
%\bibliography{combined}


\end{document}
